
\documentclass[%
a4paper,
%twoside,
11pt
]{article}

% encoding, font, language
\usepackage[T1]{fontenc}
\usepackage[latin1]{inputenc}
\usepackage{lmodern}
\usepackage[ngerman, english]{babel}

\usepackage{nicefrac}

\usepackage[nowarnings,]{xcookybooky}

\usepackage{blindtext}    % only needed for generating test text

\DeclareRobustCommand{\textcelcius}{\ensuremath{^{\circ}\mathrm{C}}}


\setcounter{secnumdepth}{1}
\renewcommand*{\recipesection}[2][]
{%
    \subsection[#1]{#2}
}
\renewcommand{\subsectionmark}[1]
{% no implementation to display the section name instead
}


\usepackage{hyperref}    % must be the last package
\hypersetup{%
    pdfauthor            = {Sven Harder},
    pdftitle             = {Example Recipes for xcookybooky},
    pdfsubject           = {Recipes},
    pdfkeywords          = {example, recipes, cookbook, xcookybooky},
    pdfstartview         = {FitV},
    pdfview              = {FitH},
    pdfpagemode          = {UseNone}, % Options; UseNone, UseOutlines
    bookmarksopen        = {true},
    pdfpagetransition    = {Glitter},
    colorlinks           = {true},
    linkcolor            = {black},
    urlcolor             = {blue},
    citecolor            = {black},
    filecolor            = {black},
}

\hbadness=10000 % Ignore underfull boxes

\begin{document}

\title{Examples for using \textbf{xcookybooky}}
\author{Sven Harder\\ \href{mailto:sven\_one1@gmx.de}{sven\_one1@gmx.de}}
\maketitle

\begin{abstract}
    \noindent The examples in this document require at least version~1.4 of the \texttt{xcookybooky}\footnote{\url{http://www.ctan.org/pkg/xcookybooky}} package. For more examples and test recipes especially for using hook functions take a look at the source files located at \url{https://code.google.com/p/xcookybooky/}. If you are interested in modifying the layout of \texttt{xcookybooky} you will find examples in the documentation as well as in the configuration file \textbf{xcookybooky.cfg}.
\end{abstract}

\tableofcontents

\vspace{5em}

\section{Recipes}
The following recipes are examples for the usage of the \texttt{xcookybooky} package. The copyright of the pictures is owned by Roman Gaus. If you are using MiKTeX~2.9 you should get no errors, no warnings and no overfull boxes. The underfull boxes are suppressed due to the settings.


\begin{recipe}
[%
    preparationtime = {\unit[1]{h}},
    bakingtime={\unit[1]{h}},
    bakingtemperature={\protect\bakingtemperature{
        fanoven=\unit[230]{\textcelcius},
        topbottomheat=\unit[195]{°C},
        topheat=\unit[195]{°C},
        gasstove=Level 2}},
    portion = {\portion{5-6}},
    calory={\unit[3]{kJ}},
    source = {Somebody you used know}
]
{Test Recipe}

    \introduction{%
        \blindtext
    }

    \ingredients
    )\\
        3 & Eggs\\
        \unit[200]{ml} & Cream\\
        40 g & Sugar\\
        50 g & Butter
    }

    \preparation
    {%
        \step \blindtext
        \step \blindtext
        \step \blindtext
    }

    \suggestion[Headline]
    {%
        \blindtext
    }

    \suggestion{%
        \blindtext
    }

    \hint{%
        Enjoy typesetting recipes with {\textbf{\Large\LaTeX}} and {\textbf{\Large xcookybooky!}}
    }

\end{recipe}

\end{document}
