
\documentclass[11pt]{report}

% encoding, font, language
\usepackage[T1]{fontenc}
\usepackage[latin1]{inputenc}
\usepackage{lmodern}
\usepackage{nicefrac}
\usepackage[nowarnings,]{xcookybooky}
\usepackage{blindtext}    % only needed for generating test text
\usepackage{verbatim}
\usepackage{hyperref}    % must be the last package
\DeclareRobustCommand{\textfar}{\ensuremath{^{\circ}\mathrm{F}}}


\setcounter{secnumdepth}{1}
\renewcommand*{\recipesection}[2][]{
    \subsection[#1]{#2}
}
\renewcommand{\subsectionmark}[1]{% no implementation to display the section name instead
}

\hypersetup{
    pdfauthor={Adam M. Resnick},
    pdftitle={Recipe Collection},
    pdfsubject={Recipes},
    pdfkeywords={example, recipes, cookbook, xcookybooky},
    pdfstartview={FitV},
    pdfview={FitH},
    pdfpagemode={UseNone}, % Options; UseNone, UseOutlines
    bookmarksopen={true},
    pdfpagetransition={Glitter},
    colorlinks={true},
    linkcolor={black},
    urlcolor={blue},
    citecolor={black},
    filecolor={black},
}
\hbadness=10000 % Ignore underfull boxes

\title{\bf{Recipe Collection}}
\author{Adam M. Resnick}
\date{\today}

\begin{document}
\maketitle

\tableofcontents

\begin{comment}
\begin{recipe}
[%
    preparationtime = {\unit[30]{min}},
    bakingtime={\unit[1]{h}},
    bakingtemperature={\protect\bakingtemperature{
        fanoven=\unit[230]{\textfar},
        topbottomheat=\unit[195]{°C},
        topheat=\unit[195]{°C},
        gasstove=Level 2}},
    portion = {\portion{5-6}},
    calory={\unit[3]{kJ}},
    source = {Somebody you used know}
]
{Test Recipe}

    \introduction{
        \blindtext
    }

    \ingredients{
        2 bar & Dark Chocolate (above \unit[70]{\%})\\
        3 & Eggs\\
        \unit[200]{ml} & Cream\\
        40 g & Sugar\\
        50 g & Butter
    }

    \preparation{
        \step \blindtext
        \step \blindtext
        \step \blindtext
    }

    \suggestion[Headline]{
        \blindtext
    }

    \suggestion{
        \blindtext
    }

    \hint{
        Enjoy typesetting recipes with {\textbf{\Large\LaTeX}} and {\textbf{\Large xcookybooky!}}
    }

\end{recipe}
\end{comment}

\chapter {Greece}

\chapter{Japan}
\section{Fish}


\begin{recipe}
[%
    preparationtime = {\unit[10-20]{min}},
    bakingtime={\unit[20]{min}},
    portion = {\portion{2}},
    source = {\href{http://kanakoskitchen.com/2010/07/19/saba-nitsuke/}{Kanako's Kitchen}}
]
{Saba Nitsuke}

    \introduction{%
        Nitsuke is a very simple simmering technique that yields a deep, sweet, salty, gingery, umami main dish in just a few minutes. A mainstay of everyday Japanese home cooking, Saba Nitsuke is definitely one of the three or four most often-cooked Japanese dishes.

        This simmering technique, by the way, works well not only with mackerel but also with just about any kind of fatty fish, including flatfish, sea bream, sardines and pacific saury.
    }

    \ingredients{
        1 & Whole mackerel (or fillets)\\
        \unit[1/3]{Cup} & Sake\\
        \unit[1/3]{Cup} & Mirin\\
        \unit[1]{Cup} & Water\\
        \unit[1/3]{Cup} & Soy Sauce\\
        \unit[3]{Tbsp} & Sugar\\
        1 & Thin ginger slice
    }

    \preparation{
        \step Clean and fillet the fish, if it is whole.
        \step Boil the sake and mirin in a pan big enough to comfortably hold the fillets without them piling up.
        \step Add water, sugar, soy sauce and ginger. Bring the mixture back to a boil.
        \step With the liquid boiling, place the fillets in the pan. Occasionally baste the fish during this time.
        \step When the liquid boils for the third time, place an otoshibuta (a lid that is slightly too small for the pan) over the fish and cook for 4 minutes.
        \step Remove the otoshibuta, and continue to simmer for another 8-10 minutes while the sauce reduces. Baste occasionally, being sure not to move the fillets, as they will break apart.
        \step The fillets should be glazed golden brown and the sauce should have reduced somewhat. It is now done
    }

    \suggestion{%
        A quick meal can be made of this, miso soup, rice and some type of pickle or rice seasoning.
    }

    \hint{%
        Cooking more fish with this recipe is really more of a function of pan size. If the pan is such that you can get two mackerel mostly in the liquid, there's no need to double the liquid portions.
    }

\end{recipe}

\end{document}
