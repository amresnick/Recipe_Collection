\documentclass[14pt]{report}

\usepackage[backend=biber, sorting=none]{biblatex}
\usepackage{units}
\usepackage{verbatim}
\usepackage[nowarnings,]{xcookybooky}
\usepackage{hyperref}    % must be the last package
\DeclareRobustCommand{\textfar}{\ensuremath{^{\circ}\mathrm{F}}}


\setcounter{secnumdepth}{1}
\renewcommand*{\recipesection}[2][]{
    \subsection[#1]{#2}
}
\renewcommand{\subsectionmark}[1]{% no implementation to display the section name instead
}

\addbibresource{../Bibtex/bibtex.bib}

\hypersetup{
    pdfauthor={Adam M. Resnick},
    pdftitle={Recipe Collection},
    pdfsubject={Recipes},
    pdfkeywords={example, recipes, cookbook, xcookybooky},
    pdfstartview={FitV},
    pdfview={FitH},
    pdfpagemode={UseNone}, % Options; UseNone, UseOutlines
    bookmarksopen={true},
    pdfpagetransition={Glitter},
    colorlinks={true},
    linkcolor={red},
    urlcolor={blue},
    citecolor={red},
    filecolor={black},
}
\hbadness=10000 % Ignore underfull boxes

\title{\bf{Recipe Collection}}
\author{Adam M. Resnick}
\date{\today}

\begin{document}
\maketitle

\tableofcontents

\begin{comment}
\begin{recipe}[%
    preparationtime = {\unit[30]{min}},
    bakingtime={\unit[1]{h}},
    bakingtemperature={\protect\bakingtemperature{
        fanoven=\unit[230]{\textfar},
        topbottomheat=\unit[195]{°C},
        topheat=\unit[195]{°C},
        gasstove=Level 2}},
    portion = {\portion{5-6}},
    calory={\unit[3]{kJ}},
    source = {Somebody you used know}
]
{Test Recipe}

    \introduction{
        \blindtext
    }

    \ingredients{
        2 bar & Dark Chocolate (above \unit[70]{\%})\\
        3 & Eggs\\
        \unit[200]{ml} & Cream\\
        40 g & Sugar\\
        50 g & Butter
    }

    \preparation{
        \step \blindtext
        \step \blindtext
        \step \blindtext
    }

    \suggestion[Headline]{
        \blindtext
    }

    \suggestion{
        \blindtext
    }

    \hint{
        Enjoy typesetting recipes with {\textbf{\Large\LaTeX}} and {\textbf{\Large xcookybooky!}}
    }

\end{recipe}
\end{comment}

% JAPAN
\chapter{Japan}

% DASHI
\section{Soup and Dashi}
\begin{recipe}[
    preparationtime = {\unit[5]{min}},
    bakingtime={\unit[15]{min}},
    portion = {\unit[4]{Cups}},
    source = {\cite{JustHungryDashi}, \cite{JustOneDashi}}
]
{Ichiban Awase Dashi}

    \introduction{%
        Awase dashi seems to be the most commonly used dashi. It's simple, and quick to make, consisting only of kombu seaweed and dried bonito flakes. Dashi is widely used in Japanese recipes, and is much faster to make than western stocks. I'm pretty loose with the measurements for this, as everyone has their own, slightly different version.
    }

    \ingredients{
        \unit[4]{Cups} & Water\\
        \unit[15]{g} & Kombu (15-20 sq. in)\\
        \unit[15-30]{g} & Katsuobushi (A good handful or two)
    }

    \preparation{
        \step Put the kombu and water in the pot. Let the kombu soak for 30 minutes to a day if possible. If you don't have time, skip the soaking step.
        \step Bring the water to a gentle boil. Just as it starts to boil (bubbles forming on the edge of the pot) take out the kombu and save it for \hyperref[recipe:NibanDashi]{Niban Dashi}. Turn off the heat.
        \step Add the katsuobushi and let it sit for 10-15 minutes. The katsuobushi flakes should mostly sink to the bottom.
        \step Strain the dashi, and freeze the katsuobushi for \hyperref[recipe:NibanDashi]{Niban Dashi}.
    }

\label{recipe:IchibanAwaseDashi}
\end{recipe}

\begin{recipe}[
    preparationtime = {\unit[10]{min}},
    bakingtime={\unit[15]{min}},
    portion = {\unit[4]{Cups}},
    source = {\cite{JustOneIrikoDashi}}
]
{Iriko  Dashi}

    \introduction{%
        Iriko dashi is made from dried anchovies, which are generally cheaper than katsuobushi. The aroma and flavor a bit more fishy than normal \hyperref[recipe:IchibanAwaseDashi]{awase dashi}. Iriko dashi seems to mainly be used to make miso soup, which is naturally strongly flavored from the miso.
    }

    \ingredients{
        \unit[4]{Cups} & Water\\
        \unit[40]{g} & Iriko (about a cup)
    }

    \preparation{
        \step If you are bored, you can decapitate and de-stomach the iriko. Supposedly, this makes the dashi less bitter. I've never tried.
        \step Soak the iriko in the water for 30 minutes to overnight.
        \step Transfer to a pot, and slowly bring to a boil. Once boiling, reduce heat to low and simmer for 10 minutes.
        \step Turn off the heat and strain the dashi.
    }

    \hint{
        Refrigerate or freeze unused dashi. It will keep for about 3 or 4 days in the fridge.
    }
\label{recipe:IrikoDashi}
\end{recipe}

\begin{recipe}[
    preparationtime = {\unit[2]{min}},
    bakingtime={\unit[25]{min}},
    portion = {\unit[4]{Cups}},
    source = {\cite{JustHungryDashi}}
]
{Niban Awase Dashi}

    \introduction{%
        Niban dashi will, of course, have a flavor that is not as strong as ichiban dashi. Thus, it is good to use it in dishes that will have their own, stronger flavor.
    }

    \ingredients{
        \unit[4]{Cups} & Water (or however much you need)\\
        1 & Set used ichiban dashi ingredients
    }

    \preparation{
        \step Put the used kombu and katsuobushi in a pot with the water. I typically use as much water as niban dashi that I will need. Bring it all to a boil on high heat.
        \step Turn down the heat and let it simmer for 10-15 minutes.
        \step Turn off the heat. You can add an extra handful of katsuobushi flakes now if you'd like it to be a bit stronger.
        \step Once the katsuobushi flakes have sunk to the bottom, strain the dashi. Discard the now twice-used ingredients.
    }

\label{recipe:NibanAwaseDashi}
\end{recipe}

\newpage
\section{Fish}

\begin{recipe}[
    preparationtime = {\unit[10]{min}},
    bakingtime={\unit[20]{min}},
    portion = {\portion{2}},
    source = {\cite{JustOneSabaMisoni}}
]
{Saba Misoni}

    \introduction{%
        Mackerel fillets are simmered in a miso-based sauce. It's an easy and quick way to cook mackerel and similar fish. This similar to \hyperref[recipe:SabaMisoni]{Saba Nitsuke}, although the flavor profile is different.
    }

    \ingredients{
        1 & Whole mackerel (or 2 fillets)\\
        \unit[\nicefrac{1}{2}]{Tsp} & Soy Sauce\\
        1 & Skinny green onion (garnish)\\
        2 & Round, decoratively cut carrot slices (garnish)\\
        \unit[2]{Tbsp} & Sake\\
        \unit[2]{Tbsp} & Sugar\\
        \unit[2]{Tbsp} & Miso\\
        \unit[\nicefrac{1}{2}]{Cup} & Water\\
        1-3 & Thin ginger slices
    }

    \preparation{
        \step Clean and fillet the fish, if it is whole. If desired, an "X" can be cut into the skin side of the fish
        \step Combine the sake, sugar, miso, ginger and water in a pan big enough to hold all the fillets. Stir well, so as to dissolve the miso. Bring the mixture to a boil.
        \step Once boiling, put fish in, skin side up. Put an otoshibuta (a slightly too small lid) on and let simmer for 8-10 minutes.
        \step Add the soy sauce, mix by swirling the pan. Take it off the heat and serve.
    }

\label{recipe:SabaMisoni}
\end{recipe}

\begin{recipe}[
    preparationtime = {\unit[10-20]{min}},
    bakingtime={\unit[20]{min}},
    portion = {\portion{2}},
    source = {\cite{KanakoSabaNitsuke}}
]
{Saba Nitsuke}

    \introduction{%
        Nitsuke is a very simple simmering technique that yields a deep, sweet, salty, gingery, umami main dish in just a few minutes. A mainstay of everyday Japanese home cooking, Saba Nitsuke is definitely one of the three or four most often-cooked Japanese dishes.

        This simmering technique, by the way, works well not only with mackerel but also with just about any kind of fatty fish, including flatfish, sea bream, sardines and pacific saury.
    }

    \ingredients{
        1 & Whole mackerel (or fillets)\\
        \unit[\nicefrac{1}{3}]{Cup} & Sake\\
        \unit[\nicefrac{1}{3}]{Cup} & Mirin\\
        \unit[1]{Cup} & Water\\
        \unit[\nicefrac{1}{3}]{Cup} & Soy Sauce\\
        \unit[3]{Tbsp} & Sugar\\
        1 & Thin ginger slice
    }

    \preparation{
        \step Clean and fillet the fish, if it is whole.
        \step Boil the sake and mirin in a pan big enough to comfortably hold the fillets without them piling up.
        \step Add water, sugar, soy sauce and ginger. Bring the mixture back to a boil.
        \step With the liquid boiling, place the fillets in the pan, skin side up. Occasionally baste the fish during this time.
        \step When the liquid boils for the third time, place an otoshibuta (a lid that is slightly too small for the pan) over the fish and cook for 4 minutes.
        \step Remove the otoshibuta, and continue to simmer for another 8-10 minutes while the sauce reduces. Baste occasionally, being sure not to move the fillets, as they will break apart.
        \step The fillets should be glazed golden brown and the sauce should have reduced somewhat. It is now done.
    }

    \suggestion{%
        A quick meal can be made of this, miso soup, rice and some type of pickle or rice seasoning.
    }

    \hint{%
        Cooking more fish with this recipe is really more of a function of pan size. If the pan is such that you can get two mackerel mostly in the liquid, there's no need to double the liquid portions.
    }

\label{recipe:SabaNitsuke}
\end{recipe}

%THE MEDITERRANEAN
\chapter{The Mediterranean}

% MEAT
\section{Meat}

\begin{recipe}[
    preparationtime = {\unit[30]{min}},
    bakingtime={\unit[2]{hours}},
    portion = {\unit[4]{Servings}},
    source = {Adapted from \cite[p. 237]{EssentialsOfMediterraneanCooking}}
]
{Pork Smothered in Leeks}

    \introduction{%
        Slowly braising the pork in this Greek recipe results in very tender meat. Nine leeks might sound like a lot, but by the time the dish finishes braising, the leek will not be overpowering.
    }

    \ingredients{
        \unit[1.75]{lb} & Pork butt cut into 1 inch cubes\\
        \unit[1]{Tbsp} & Olive oil\\
        1 & Chopped yellow onion\\
        \unit[1]{Cup} & Chopped tomatoes\\
        1 & Chopped celery stalk\\
        9 & Leek stalks
    }

    \preparation{
        \step Sprinkle meat with sea salt and fresh ground pepper.
        \step Put the olive oil in a pot over medium heat. Once hot, brown the meat, turning to brown all sides. This will take 8-10 minutes. If working in groups, put the browned meat on a plate and move on to the next batch of meat.
        \step Return all the meat to the pot, add the onion and cook for about 5 minutes, until the onion has softened.
        \step Add the tomatoes (fresh or canned), celery and enough water to almost cover all the meat. Bring to a boil, stirring occasionally.
        \step Reduce the heat to medium-low, cover and simmer gently until the meat is starting to be tender, about 45 minutes or an hour.
        \step Halve the leeks lengthwise and rinse thoroughly as there is sometimes sand between the layers of the leek.
        \step Put the cleaned leeks in a pan, add a small layer of water, bring it to a boil and reduce to medium-low. Simmer for 10 minutes, then drain. Alternatively, saut\'e in a covered pan with olive oil.
        \step Put the leeks over the pork, cover and simmer until most of the liquid has been absorbed and the meat and leeks are very tender (another 30 minutes to an hour).
    }

\label{recipe:PorkSmotheredInLeeks}
\end{recipe}


\section{Vegetables}
\begin{recipe}[
    preparationtime = {\unit[10]{min}},
    bakingtime={\unit[15]{min}},
    portion = {\portion{4}},
    source = Adapted from {\cite[p. 151]{EssentialsOfMediterraneanCooking}}
]
{Swiss Chard with Raisins and Pine Nuts}

    \introduction{%
        Originating from Spain, this is a quick way to prepare swiss chard. The original recipe calls for Serrano ham to be chopped and cooked with the swiss chard. I don't usually have ham sitting at home though, so I've removed it.
    }

    \ingredients{
        \unit[\nicefrac{1}{4}]{Cup} & Golden raisins\\
        \unit[3]{Tbsp} & Extra-virgin olive oil\\
        \unit[3]{Tbsp} & Pine Nuts\\
        \unit[1\nicefrac{3}{4}]{lb} & Swiss chard\\
        2 & Garlic cloves, chopped
    }

    \preparation{
        \step Over medium heat, warm a tablespoon of olive oil. Add the pine nuts, and toast for 2-3 minutes, until the nuts are just becoming golden. It helps to swirl the pan quite a bit to keep the nuts from burning on any one side. Set the nuts aside.
        \step Cut off the chard stems and save them for something else. Cut the leaves into one inch strips.
        \step Heat the remaining olive oil over medium-high. Add the swiss chard and garlic, stirring to work the olive oil in. Cover the pan, stirring occasionally to ensure even cooking of the chard. Continue until the leaves are all wilted.
        \step Add the raisins, and cook uncovered for another 5 minutes. Add the pine nuts, salt and pepper (if desired), stir and remove from heat.
    }
\end{recipe}

\printbibliography[heading=bibintoc]
\end{document}
