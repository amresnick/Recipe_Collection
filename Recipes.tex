
\documentclass[11pt]{report}

% encoding, font, language
\usepackage[T1]{fontenc}
\usepackage[latin1]{inputenc}
\usepackage{lmodern}
\usepackage{nicefrac}
\usepackage[nowarnings,]{xcookybooky}
\usepackage{blindtext}    % only needed for generating test text
\usepackage{verbatim}
\usepackage{hyperref}    % must be the last package
\DeclareRobustCommand{\textfar}{\ensuremath{^{\circ}\mathrm{F}}}


\setcounter{secnumdepth}{1}
\renewcommand*{\recipesection}[2][]{
    \subsection[#1]{#2}
}
\renewcommand{\subsectionmark}[1]{% no implementation to display the section name instead
}

\hypersetup{
    pdfauthor={Adam M. Resnick},
    pdftitle={Recipe Collection},
    pdfsubject={Recipes},
    pdfkeywords={example, recipes, cookbook, xcookybooky},
    pdfstartview={FitV},
    pdfview={FitH},
    pdfpagemode={UseNone}, % Options; UseNone, UseOutlines
    bookmarksopen={true},
    pdfpagetransition={Glitter},
    colorlinks={true},
    linkcolor={black},
    urlcolor={blue},
    citecolor={black},
    filecolor={black},
}
\hbadness=10000 % Ignore underfull boxes

\title{\bf{Recipe Collection}}
\author{Adam M. Resnick}
\date{\today}

\begin{document}
\maketitle

\tableofcontents

\begin{comment}
\begin{recipe}
[%
    preparationtime = {\unit[30]{min}},
    bakingtime={\unit[1]{h}},
    bakingtemperature={\protect\bakingtemperature{
        fanoven=\unit[230]{\textfar},
        topbottomheat=\unit[195]{°C},
        topheat=\unit[195]{°C},
        gasstove=Level 2}},
    portion = {\portion{5-6}},
    calory={\unit[3]{kJ}},
    source = {Somebody you used know}
]
{Test Recipe}

    \introduction{%
        \blindtext
    }

    \ingredients
    )\\
        3 & Eggs\\
        \unit[200]{ml} & Cream\\
        40 g & Sugar\\
        50 g & Butter
    }

    \preparation
    {%
        \step \blindtext
        \step \blindtext
        \step \blindtext
    }

    \suggestion[Headline]
    {%
        \blindtext
    }

    \suggestion{%
        \blindtext
    }

    \hint{%
        Enjoy typesetting recipes with {\textbf{\Large\LaTeX}} and {\textbf{\Large xcookybooky!}}
    }

\end{recipe}
\end{comment}

\chapter {Greece}

\chapter{Japan}
\section{Fish}


\begin{recipe}
[%
    preparationtime = {\unit[20]{min}},
    bakingtime={\unit[15]{min}},
    portion = {\portion{4}},
    source = {http://kanakoskitchen.com/2010/07/19/saba-nitsuke/}
]
{Saba Nitsuke}

    \introduction{%
        Nitsuke is a very simple simmering technique that yields a deep, sweet, salty, gingery, umami main dish in just a few minutes. A mainstay of everyday Japanese home cooking, Saba Nitsuke is definitely one of the three or four most often-cooked Japanese dishes.

        This simmering technique, by the way, works well not only with mackerel but also with just about any kind of fatty fish, including flatfish, sea bream, sardines and pacific saury, too.
    }

    \ingredients
    {%
        1 & whole mackerel (or fillets)\\
        \unit[1/3]{cup} & Sake\\
        \unit[1/3]{cup} & Mirin\\
        \unit[1]{cup} & Water\\
        \unit[1/3]{cup} & Soy Sauce\\
        \unit[3]{tablespoons} & Sugar\\
        1 & Thinly sliced ginger piece
    }

    \preparation
    {%
        \step \blindtext
        \step \blindtext
        \step \blindtext
    }

    \suggestion[Headline]
    {%
        \blindtext
    }

    \suggestion{%
        \blindtext
    }

    \hint{%
        Enjoy typesetting recipes with {\textbf{\Large\LaTeX}} and {\textbf{\Large xcookybooky!}}
    }

\end{recipe}

\end{document}
