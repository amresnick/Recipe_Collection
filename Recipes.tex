\documentclass{report}

\usepackage[backend=biber, sorting=none]{biblatex}
\usepackage{units}
\usepackage{verbatim}
\usepackage[nowarnings,]{xcookybooky}
\usepackage{hyperref}    % must be the last package
\DeclareRobustCommand{\textfar}{\ensuremath{^{\circ}\mathrm{F}}}


\setcounter{secnumdepth}{1}
\renewcommand*{\recipesection}[2][]{
    \subsection[#1]{#2}
}
\renewcommand{\subsectionmark}[1]{% no implementation to display the section name instead
}

\addbibresource{../Bibtex/bibtex.bib}

\hypersetup{
    pdfauthor={Adam M. Resnick},
    pdftitle={Recipe Collection},
    pdfsubject={Recipes},
    pdfkeywords={recipes, cookbook, xcookybooky},
    pdfstartview={FitV},
    pdfview={FitH},
    pdfpagemode={UseNone}, % Options; UseNone, UseOutlines
    bookmarksopen={true},
    pdfpagetransition={Glitter},
    colorlinks={true},
    linkcolor={red},
    urlcolor={blue},
    citecolor={red},
    filecolor={black},
}
\hbadness=10000 % Ignore underfull boxes

\title{\bf{Recipe Collection}}
\author{Adam M. Resnick}
\date{\today}

\begin{document}
\maketitle

\tableofcontents

\begin{comment}
\begin{recipe}[%
    preparationtime = {\unit[30]{min}},
    bakingtime={\unit[1]{h}},
    bakingtemperature={\protect\bakingtemperature{
        fanoven=\unit[230]{\textfar},
        topbottomheat=\unit[195]{°C},
        topheat=\unit[195]{°C},
        gasstove=Level 2}},
    portion = {\portion{5-6}},
    calory={\unit[3]{kJ}},
    source = {Somebody you used know}
]
{Test Recipe}

    \introduction{
        \blindtext
    }

    \ingredients{
        2 bar & Dark Chocolate (above \unit[70]{\%})\\
        3 & Eggs\\
        \unit[200]{ml} & Cream\\
        40 g & Sugar\\
        50 g & Butter
    }

    \preparation{
        \step \blindtext
        \step \blindtext
        \step \blindtext
    }

    \suggestion[Headline]{
        \blindtext
    }

    \suggestion{
        \blindtext
    }

    \hint{
        Enjoy typesetting recipes with {\textbf{\Large\LaTeX}} and {\textbf{\Large xcookybooky!}}
    }

\end{recipe}
\end{comment}

% JAPAN
\chapter{Japan}

% DASHI
\section{Soup and Dashi}
\begin{recipe}[
    preparationtime = {\unit[5]{min}},
    bakingtime={\unit[15]{min}},
    portion = {\unit[4]{Cups}},
    source = {\cite{JustHungryDashi}, \cite{JustOneDashi}}
]
{Ichiban Awase Dashi}

    \introduction{%
        Awase dashi seems to be the most commonly used dashi. It's simple, and quick to make, consisting only of kombu seaweed and dried bonito flakes. Dashi is widely used in Japanese recipes, and is much faster to make than western stocks. I'm pretty loose with the measurements for this, as everyone has their own, slightly different version.
    }

    \ingredients{
        \unit[4]{Cups} & Water\\
        \unit[15]{g} & Kombu (15-20 sq. in)\\
        \unit[15-30]{g} & Katsuobushi (A good handful or two)
    }

    \preparation{
        \step Put the kombu and water in the pot. Let the kombu soak for 30 minutes to a day if possible. If you don't have time, skip the soaking step.
        \step Bring the water to a gentle boil. Just as it starts to boil (bubbles forming on the edge of the pot) take out the kombu and save it for \hyperref[recipe:NibanAwaseDashi]{Niban Dashi}. Turn off the heat.
        \step Add the katsuobushi and let it sit for 10-15 minutes. The katsuobushi flakes should mostly sink to the bottom.
        \step Strain the dashi, and freeze the katsuobushi for \hyperref[recipe:NibanAwaseDashi]{Niban Dashi}.
    }

\label{recipe:IchibanAwaseDashi}
\end{recipe}

\begin{recipe}[
    preparationtime = {\unit[10]{min}},
    bakingtime={\unit[15]{min}},
    portion = {\unit[4]{Cups}},
    source = {\cite{JustOneIrikoDashi}}
]
{Iriko  Dashi}

    \introduction{%
        Iriko dashi is made from dried anchovies, which are generally cheaper than katsuobushi. The aroma and flavor a bit more fishy than normal \hyperref[recipe:IchibanAwaseDashi]{awase dashi}. Iriko dashi seems to mainly be used to make miso soup, which is naturally strongly flavored from the miso.
    }

    \ingredients{
        \unit[4]{Cups} & Water\\
        \unit[40]{g} & Iriko (about a cup)
    }

    \preparation{
        \step If you are bored, you can decapitate and de-stomach the iriko. Supposedly, this makes the dashi less bitter. I've never tried.
        \step Soak the iriko in the water for 30 minutes to overnight.
        \step Transfer to a pot, and slowly bring to a boil. Once boiling, reduce heat to low and simmer for 10 minutes.
        \step Turn off the heat and strain the dashi.
    }

    \hint{
        Refrigerate or freeze unused dashi. It will keep for about 3 or 4 days in the fridge.
    }
\label{recipe:IrikoDashi}
\end{recipe}

\begin{recipe}[
    preparationtime = {\unit[2]{min}},
    bakingtime={\unit[25]{min}},
    portion = {\unit[4]{Cups}},
    source = {\cite{JustHungryDashi}}
]
{Niban Awase Dashi}

    \introduction{%
        Niban dashi will, of course, have a flavor that is not as strong as ichiban dashi. Thus, it is good to use it in dishes that will have their own, stronger flavor.
    }

    \ingredients{
        \unit[4]{Cups} & Water (or however much you need)\\
        1 & Set used ichiban dashi ingredients
    }

    \preparation{
        \step Put the used kombu and katsuobushi in a pot with the water. I typically use as much water as niban dashi that I will need. Bring it all to a boil on high heat.
        \step Turn down the heat and let it simmer for 10-15 minutes.
        \step Turn off the heat. You can add an extra handful of katsuobushi flakes now if you'd like it to be a bit stronger.
        \step Once the katsuobushi flakes have sunk to the bottom, strain the dashi. Discard the now twice-used ingredients.
    }

\label{recipe:NibanAwaseDashi}
\end{recipe}

%FISH
\newpage
\section{Fish}

\begin{recipe}[
    preparationtime = {\unit[10]{min}},
    bakingtime={\unit[20]{min}},
    portion = {\unit[2]{Servings} },
    source = {\cite{JustOneSabaMisoni}}
]
{Saba Misoni}

    \introduction{%
        Mackerel fillets are simmered in a miso-based sauce. It's an easy and quick way to cook mackerel and similar fish. This similar to \hyperref[recipe:SabaMisoni]{Saba Nitsuke}, although the flavor profile is different. If you can fit more mackerel in the pan and have the fillets about halfway submerged, then there's no need to double the sauce ingredients - just put in more fish, if you have any more.

        The prep time depends on how fast you can clean and fillet a whole fish.
    }

    \ingredients{
        1 & Whole mackerel (or 2 fillets)\\
        \unit[\nicefrac{1}{2}]{Tsp} & Soy Sauce\\
        1 & Skinny green onion (garnish)\\
        2 & Round, decoratively cut carrot slices (garnish)\\
        \unit[2]{Tbsp} & Sake\\
        \unit[2]{Tbsp} & Sugar\\
        \unit[2]{Tbsp} & Miso\\
        \unit[\nicefrac{1}{2}]{Cup} & Water\\
        1-3 & Thin ginger slices
    }

    \preparation{
        \step Clean and fillet the fish, if it is whole. If desired, an "X" can be cut into the skin side of the fish
        \step Combine the sake, sugar, miso, ginger and water in a pan big enough to hold all the fillets. Stir well, so as to dissolve the miso. Bring the mixture to a boil.
        \step Once boiling, put fish in, skin side up. Put an otoshibuta (a slightly too small lid) on and let simmer for 8-10 minutes.
        \step Add the soy sauce, mix by swirling the pan. Take it off the heat and serve.
    }

\label{recipe:SabaMisoni}
\end{recipe}

\begin{recipe}[
    preparationtime = {\unit[10-20]{min}},
    bakingtime={\unit[20]{min}},
    portion = {\unit[2]{Servings} },
    source = {\cite{KanakoSabaNitsuke}}
]
{Saba Nitsuke}

    \introduction{%
        Nitsuke is a very simple simmering technique that yields a deep, sweet, salty, gingery, umami main dish in just a few minutes. This simmering technique, by the way, works well not only with mackerel but also with just about any kind of fatty fish, including flatfish, sea bream, sardines and pacific saury. The prep time depends on how fast you can clean and fillet a whole fish.
    }

    \ingredients{
        1 & Whole mackerel (or fillets)\\
        \unit[\nicefrac{1}{3}]{Cup} & Sake\\
        \unit[\nicefrac{1}{3}]{Cup} & Mirin\\
        \unit[1]{Cup} & Water\\
        \unit[\nicefrac{1}{3}]{Cup} & Soy Sauce\\
        \unit[3]{Tbsp} & Sugar\\
        1 & Thin ginger slice
    }

    \preparation{
        \step Clean and fillet the fish, if it is whole.
        \step Boil the sake and mirin in a pan big enough to comfortably hold the fillets without them piling up.
        \step Add water, sugar, soy sauce and ginger. Bring the mixture back to a boil.
        \step With the liquid boiling, place the fillets in the pan, skin side up. Occasionally baste the fish during this time.
        \step When the liquid boils for the third time, place an otoshibuta (a lid that is slightly too small for the pan) over the fish and cook for 4 minutes.
        \step Remove the otoshibuta, and continue to simmer for another 8-10 minutes while the sauce reduces. Baste occasionally, being sure not to move the fillets, as they will break apart.
        \step The fillets should be glazed golden brown and the sauce should have reduced somewhat. It is now done.
    }

    \suggestion{
        A quick meal can be made of this, miso soup, rice and some type of pickle or rice seasoning.
    }

    \hint{
        Cooking more fish with this recipe is really more of a function of pan size. If the pan is such that you can get two mackerel mostly in the liquid, there's no need to double the liquid portions.
    }

\label{recipe:SabaNitsuke}
\end{recipe}

\begin{recipe}[
    preparationtime = {\unit[10-20]{min}},
    bakingtime={\unit[10]{min}},
    portion = {\unit[2]{Servings} },
    source = {\cite{JustOneSabaShioyaki}}
]
{Saba Shioyaki}

    \introduction{
        The prep time depends on the form the mackerel starts in and how fast you can clean and fillet a whole fish.

        Salt grilling (shioyaki) is a quick way to prepare any type of flavorful fish. It's pretty common in Japanese restaurants.
    }

    \ingredients{
        1 & Whole mackerel (or fillets)\\
        \unit[1]{Tbsp} & Sake\\
         & Salt
    }

    \preparation{
        \step Clean and fillet the fish, if it is whole. Pat dry the fillets.
        \step Pour the sake into a medium bowl and run the fillets through the sake before setting them aside on a plate.
        \step Sprinkle the fillet's with salt and set aside for 20-30 minutes. Start the grill or preheat the oven to 400\unit{F}.
        \step If using the oven, line a baking sheet with some parchment paper. Parchment paper can also be useful to prevent sticking and loss of skin on the grill.
        \step Cook for 7-10 minutes.
    }

\label{recipe:SabaShioyaki}
\end{recipe}

%THE MEDITERRANEAN
\chapter{The Mediterranean}

% BREAD AND PASTRIES
\section{Breads and Pastries}
\begin{recipe}[
    preparationtime = {\unit[3.5]{hours}},
    bakingtime={\unit[30-45]{min}},
    portion = {\unit[12]{Pastries}},
    source = {Adapted from \cite{SultansKitchen}}
]
{Spinach and Lentil Savory Pastries}

    \ingredients{
        \unit[4]{Cups} & All purpose flour\\
        \unit[\nicefrac{1}{2}]{Tsp} & Salt\\
        \unit[2] & Beaten eggs\\
        \unit[1]{Tbsp} & White wine vinegar\\
        \unit[1]{Cup} & Water\\
        \unit[\nicefrac{3}{4}]{Cup} & Green lentils\\
        \unit[\nicefrac{3}{2}]{Cup} & Water\\
        \unit[3] & Tomatoes\\
        \unit[1]{Bunch} & Scallions\\
        \unit[1] & Medium onion\\
        \unit[1.5]{lbs} & Raw spinach\\
        & Olive oil\\
        \unit[1] & Egg\\
        \unit[2]{Tbsp} & Milk\\
        \unit[\nicefrac{1}{2}]{Stick} & Chilled, unsalted butter
    }

    \preparation{
        \step To make the dough, mix the flour and salt in a large bowl, or directly on a counter top. Make a well in the middle and pour in the beaten eggs, white wine vinegar and the cup of water. Mix together until the dough begins to form a ball, then kneed for 8-10 minutes. Form the dough into a cube. Wrap the dough cube in plastic wrap, or cover it in a bowl and place it in the fridge for an hour to cool down.
        \step While the dough chills down, begin making the stuffing by washing the lentils and placing them in a pot. Add the 1.5 cups of water, cover the pot and heat until the water has just started to boil. Then, turn the heat down to low or medium low and simmer for 20-30 minutes.
        \step As the lentils finish cooking, put some olive oil into a large pan and heat over medium heat. Peel and dice the tomatoes. Add the onion and scallion to the pan and cook for 5 minutes or so. Add in the tomato, cook for a few minutes and begin adding the spinach. The spinach may have to be added in several batches depending on pot size. Once all of the spinach has wilted, turn off the heat, salt to taste and let cool.
        \step As the filling cools down, take the half stick of butter, smash it down and roll it into about a 6 inch square. I typically do this on a plastic cutting board or a piece of parchment paper. After the butter has been rolled out, put it back in the refrigerator.
        \step Take the dough cube out of the refrigerator. Lightly flour the top and bottom of the dough, as well as the countertop. Roll the dough out into about an 18 inch square. Place the butter in the middle of the dough and fold the edges of the dough in. You should now have a layered square of dough. Depending on the room temperature, you may now need to cool the dough down again in the refrigerator or freezer. The dough will get difficult to work with if it gets too warm due to the butter.
        \step Ensure the countertop is still floured and then roll the dough again into another 18 inch square. Divide the dough into 12 roughly equal rectangles using a dough scraper or the back of a knife. I typically divide the dough into thirds, and then divide each third into fourths.
        \step Divide the stuffing roughly equally into each piece of dough. Fold two edges of the pastry towards the center and slightly overlap them. Pinch the unfolded edges of the dough together to seal and lightly press the center seam together. Preheat the oven to 350$^{\circ}$F and place a sheet of parchment paper onto a baking sheet. Transfer the pastries to the baking sheet.
        \step Beat the milk and egg together in a small bowl. Brush the egg wash over the top of the pastries. Bake for 30-45 minutes, or until golden brown on top. Remove from oven, and cool on a wire rack.
    }

\label{recipe:SpinachLentilSavoryPastries}
\end{recipe}

% SALADS
\newpage
\section{Salads}
\begin{recipe}[
    preparationtime = {\unit[30]{min}},
    portion = {\unit[1]{Serving}},
    source = {Adapted from \cite[p. 86]{ItsAllGreekToMe}}
]
{Greek Village Salad (Horiatki Salata)}

    \ingredients{
        \unit[5]{} & Ripe tomatoes\\
        \unit[1]{} & Bell pepper\\
        \unit[5-10] & Kalamata olives\\
        \unit[2-3]{ slices} & Brined feta cheese\\
        \unit[1]{Tsp} & Dried oregano\\
        \unit[1]{} & Lemon (juiced)\\
        \unit[\nicefrac{1}{3}]{Cup} & Extra-virgin olive oil\\
        & Chopped Italian parsley\\
        & Chopped fresh dill
    }

    \preparation{
        \step Cut up the bell pepper and tomatoes and combine with the parsley, dill and olives in a large bowl. I usually save a bit of the dill for garnishing at the end.
        \step Slowly pour the olive oil into the lemon juice, whisking as you go to emulsify the oil.
        \step Drizzle the lemon juice and olive oil mixture over the salad. Toss evenly.
        \step Top with the feta slices. Crush the oregano in your hand and sprinkle it over the vegetables and the cheese. Garnish with the reserved dill.
    }

\label{recipe:GreekVillageSalad}
\end{recipe}

% MEAT
\newpage
\section{Meat}

\begin{recipe}[
    preparationtime = {\unit[30]{min}},
    bakingtime={\unit[2]{hours}},
    portion = {\unit[4]{Servings}},
    source = {Adapted from \cite[p. 237]{EssentialsOfMediterraneanCooking}}
]
{Pork Smothered in Leeks}

    \introduction{
        Slowly braising the pork in this Greek recipe results in very tender meat. Nine leeks might sound like a lot, but by the time the dish finishes braising, the leek will not be overpowering.
    }

    \ingredients{
        \unit[1.75]{lb} & Pork butt cut into 1 inch cubes\\
        \unit[1]{Tbsp} & Olive oil\\
        1 & Chopped yellow onion\\
        \unit[1]{Cup} & Chopped tomatoes\\
        1 & Chopped celery stalk\\
        9 & Leek stalks
    }

    \preparation{
        \step Sprinkle meat with sea salt and fresh ground pepper.
        \step Put the olive oil in a pot over medium heat. Once hot, brown the meat, turning to brown all sides. This will take 8-10 minutes. If working in groups, put the browned meat on a plate and move on to the next batch of meat.
        \step Return all the meat to the pot, add the onion and cook for about 5 minutes, until the onion has softened.
        \step Add the tomatoes (fresh or canned), celery and enough water to almost cover all the meat. Bring to a boil, stirring occasionally.
        \step Reduce the heat to medium-low, cover and simmer gently until the meat is starting to be tender, about 45 minutes or an hour.
        \step Halve the leeks lengthwise and rinse thoroughly as there is sometimes sand between the layers of the leek.
        \step Put the cleaned leeks in a pan, add a small layer of water, bring it to a boil and reduce to medium-low. Simmer for 10 minutes, then drain. Alternatively, saut\'e in a covered pan with olive oil.
        \step Put the leeks over the pork, cover and simmer until most of the liquid has been absorbed and the meat and leeks are very tender (another 30 minutes to an hour).
    }

\label{recipe:PorkSmotheredInLeeks}
\end{recipe}

%VEGETABLES
\newpage
\section{Vegetables}
\begin{recipe}[
    preparationtime = {\unit[10]{min}},
    bakingtime={\unit[15]{min}},
    portion = {\unit[2]{Servings} },
    source = Me
]
{Roasted Eggplant}

    \introduction{
        There are many types of eggplant available. I've taken to using what are called Asian or Chinese eggplants. They are long and skinny and tend to be less bitter once cooked compared to the other varieties available here. They're also easier to find and conveniently cheaper than other types of eggplant (if you go to a good Asian market).

        All of the recipes I've seen always specify using olive oil. I'm somewhat skeptical of this as olive oil has a pretty low smoke point and roasting is done at a pretty high temp. It might make more sense to use a higher temperature oil.
    }

    \ingredients{
        \unit[3-4]{} & Chinese/Japanese/Asian Eggplants\\
          & Extra-virgin olive oil\\
          & Salt \& pepper
    }

    \preparation{
        \step Preheat oven to 475F.
        \step Slice the eggplants and arrange them on a baking sheet lined with parchment paper.
        \step Brush the slices with olive oil, sprinkle with salt and pepper.
        \step Roast in the oven for 15 minutes. When the eggplant slices first come out of the oven, they will be crispy on the outside and soft on the inside; they are best served fresh. As they cool down, they will loose their crispiness.
    }
\label{recipe:RoastedEggplant}
\end{recipe}

\begin{recipe}[
    preparationtime = {\unit[45]{min}},
    bakingtime={\unit[40]{min}},
    portion = {\unit[6]{Servings}},
    source = {Adapted from \cite[p. 175]{EssentialsOfMediterraneanCooking}}
]
{Spiced Pumpkin Tagine}

    \ingredients{
        \unit[1]{} & Large yellow onion\\
        \unit[\nicefrac{3}{4}]{Cup} & Water\\
        \unit[1]{} & 1 Butternut squash, peeled, seeded, cut into 1 inch cubes\\
        \unit[1]{} & 1 Large carrot, peeled, cut into 0.5 inch slices\\
        \unit[1]{} & 1 Large, ripe tomato, halved, seeded, chopped\\
        \unit[1]{} & 1 Large sweet potato\\
        \unit[3]{Tbsp} & Dried currants or golden raisins\\
        \unit[1]{Tbsp} & Honey\\
        \unit[1]{Tsp} & Ground fresh ginger\\
        \unit[2]{Tbsp} & Extra-virgin olive oil\\
        \unit[\nicefrac{1}{2}]{Tsp} & Ground cinnamon\\
        \unit[\nicefrac{1}{2}]{Tsp} & Ground turmeric\\
    }

    \preparation{
        \step Heat the olive oil in a covered wok, or similar heavy covered pan.
        \step Add the onion and cook until soft, about 5 minutes.
        \step Stir in the ginger, cinnamon and turmeric and cook until the mixture becomes fragrant.
        \step Add the squash, carrot, tomato, currants, honey and water. Season to taste with salt and pepper. Bring to a boil, reduce the heat to medium, cover and simmer for 10 minutes.
        \step Peel the sweet potato, cut in half lengthwise and cut each half into \nicefrac{3}{4} inch thick slices. Add to the pot, re-cover and cook until everything is tender, about 25 minutes.
    }

\label{recipe:SpicedPumpkinTagine}
\end{recipe}

\begin{recipe}[
    preparationtime = {\unit[30]{min}},
    bakingtime= {\unit[40]{min}},
    portion = {\unit[4]{Servings}},
    source = {Adapted from \cite[p. 251]{ItsAllGreekToMe}}
]
{Stewed Zucchini with Tomatoes and Olive Oil (Kolokithia Lathera)}

    \ingredients{
        \unit[1.5]{lbs} & Cubed zucchini\\
        \unit[1]{Cup} & Extra-virgin olive oil\\
        \unit[2]{}  & Chopped yellow onions\\
        \unit[4]{}  & Ripe, peeled and wedged tomatoes\\
        \unit[1]{Tbsp} & Tomato paste, diluted in \unit[1]{C} of water\\
        \unit[3]{}  & Finely chopped garlic cloves\\
        \unit[2]{Tbsp} & Finely chopped Italian flat-leaf parsley\\
                       & Salt \& pepper
    }

    \preparation{
        \step Heat the olive oil in a pan over medium heat. Saute the onion until lightly carmelized.
        \step Add the tomatoes, tomato paste-water mixture and the garlic. Turn the heat up to medium-high, bring to a boil, then let simmer.
        \step Add the zucchini, parsley, salt and pepper. Cover and cook for 20-25 minutes or until most of the liquid has evaporated off.
    }

    \hint{
        The original recipe calls for adding another cup of water when you add the zucchini, parsley etc. I have found that I have plenty of liquid without this, but feel free to add more water. This is probably especially dependant on your pot.
    }
\label{recipe:StewedZucchini}
\end{recipe}

\begin{recipe}[
    preparationtime = {\unit[10]{min}},
    bakingtime= {\unit[15]{min}},
    portion = {\unit[2]{Servings}},
    source = {Adapted from \cite[p. 151]{EssentialsOfMediterraneanCooking}}
]
{Swiss Chard with Raisins and Pine Nuts}

    \ingredients{
        \unit[\nicefrac{1}{4}]{Cup} & Golden raisins\\
        \unit[3]{Tbsp} & Extra-virgin olive oil\\
        \unit[3]{Tbsp} & Pine Nuts\\
        \unit[1\nicefrac{3}{4}]{lb} & Swiss chard\\
        2 & Garlic cloves, chopped
    }

    \preparation{
        \step Over medium heat, warm a tablespoon of olive oil. Add the pine nuts, and toast for 2-3 minutes, until the nuts are just becoming golden. It helps to swirl the pan quite a bit to keep the nuts from burning on any one side. Set the nuts aside.
        \step Cut off the chard stems and save them for something else. Cut the leaves into one inch strips.
        \step Heat the remaining olive oil over medium-high. Add the swiss chard and garlic, stirring to work the olive oil in. Cover the pan, stirring occasionally to ensure even cooking of the chard. Continue until the leaves are all wilted.
        \step Add the raisins, and cook uncovered for another 5 minutes. Add the pine nuts, salt and pepper (if desired), stir and remove from heat.
    }
\label{recipe:SwissChardWithRaisinsPineNuts}
\end{recipe}

\printbibliography[heading=bibintoc]
\end{document}
